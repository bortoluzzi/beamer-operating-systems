\documentclass{beamer}      % This is a beamer
\usepackage[utf8]{inputenc} % Language support
%\usepackage{minted}         % Source code

\usepackage{utopia}         % Utopia Font

\usetheme{Madrid}
\usecolortheme{default}

%\graphicspath{./images/}

%%------------------------------------------------------------
%This block of code defines the information to appear in the
%Title page
\title[Operating Systems] %optional
{Operating Systems}

\subtitle{Beamer version 0.1}

\author[Fabricio Bortoluzzi] % (optional)
{Fabricio Bortoluzzi\inst{1}}

\institute[UNIVALI] % (optional)
{
  \inst{1}%
  Computer Networks Laboratory\\
  in collaboration with\\
  Laboratory of Embedded and Distributed Systems\\
}

\date[UNIVALI 2019] % (optional)
{www.univali.br\\leds.acad.univali.br}

%\logo{\includegraphics[height=1.5cm]{univali-logo.jpg}}

%End of title page configuration block
%------------------------------------------------------------



%------------------------------------------------------------
%The next block of commands puts the table of contents at the 
%beginning of each section and highlights the current section:

\AtBeginSection[]
{
  \begin{frame}
    \frametitle{Table of Contents}
    \tableofcontents[currentsection]
  \end{frame}
}
%------------------------------------------------------------


\begin{document}

%The next statement creates the title page.
\frame{\titlepage}


%---------------------------------------------------------
%This block of code is for the table of contents after
%the title page
\begin{frame}
\frametitle{Table of Contents}
\tableofcontents
\end{frame}
%---------------------------------------------------------

%------------------------------------------------------------
%Title page
\title[Operating Systems]
{Operating Systems}

\subtitle{Beamer version 0.1}

\author[Fabricio Bortoluzzi]
{Fabricio Bortoluzzi\inst{1}}

\institute[UNIVALI]
{
  \inst{1}%
  Computer Networks Laboratory\\
  in collaboration with\\
  Laboratory of Embedded and Distributed Systems\\
}

\date[UNIVALI 2019] % (optional)
{www.univali.br\\leds.acad.univali.br}

%\logo{\includegraphics[height=1.5cm]{univali-logo.jpg}}

%End of title page configuration block
%------------------------------------------------------------



%------------------------------------------------------------
%The next block of commands puts the table of contents at the 
%beginning of each section and highlights the current section:

\AtBeginSection[]
{
  \begin{frame}
    \frametitle{Table of Contents}
    \tableofcontents[currentsection]
  \end{frame}
}
%------------------------------------------------------------


\begin{document}

%The next statement creates the title page.
\frame{\titlepage}


%---------------------------------------------------------
%This block of code is for the table of contents after
%the title page
\begin{frame}
\frametitle{Table of Contents}
\tableofcontents
\end{frame}
%---------------------------------------------------------


\include{sections/1_introduction}
\include{sections/2_memory_management}
\include{sections/3_filesystems}
\include{sections/4_input_output}
\include{sections/5_deadlocks}

%\include{main-examples}

\end{document}